\section{Model}

% a precise description of your approach
In this section we present and discuss three different approaches for including the genre information in a basic seq2seq2 model. For the evaluation of these models we use an old fashioned seq2seq model as comparison, with no genre information.

\subsection{Bos\_genre}
The first approach is to replace the <bos> tag in the decoder sequence with a tag for the genre (e.g., <comedy>). These genre tags are included in the vocabulary and hence are encoded as indices, and then embedded. The advantages of this approach are that it is easy to implement, and that it is very cheap in terms of additional model parameters. Apart from the tag embeddings no parameters are added. One potential disadvantage may be the difficulty to have long range dependency. The model might "forget" the information about the genre while decoding long sentences.

\subsection{Concat\_one\_hot}
In order to tackle potentail problems with long term dependencies, this model appends a one hot encoding of the genre to the word embedding. The though behind this approach is that the genre should affect each step of the model. A disadvantage of this model might be the independence among the genres.

\subsection{Concat\_embedding}
This model embeds the genre before concatenating it to the word embeddings. This should result in the ability to model genre information. Unfortunately, the genre information is compressed further, while we likely already have a hard time modelling it.
